%!TEX root = proj_report_outline.tex
\chapter{Conclusions and Future Work}
This chapter reviews the project's contributions in light of the design requirements, discusses potential future work and finally, surmises the project.

\section{Review}

The project's principal contributions are as follows:

\paragraph{C1:} I have developed an {\em Innovation Role Taxonomy} for classifying existing innovation platforms. The relationship semantics between roles in the taxonomy is denoted as either explicitly supported or implicitly supported. Platforms that facilitate innovation collaboration ideally have explicit support for all roles.

\paragraph{C2:} I have designed a {\em new UI} to encourage collaboration and be visually appealing. Content templates and wire frames were key artefacts produced to guide the final implementation.

\paragraph{C3:} I have designed the {\em PitchHub distributed architecture}. The tiered architecture style adopted supports ease of extension for future development.

\paragraph{C4:} I have developed {\em the first suite of user stories} that capture the interactions of innovation collaboration. These stories specify the following: engagement of all roles within the innovation community, the ability to scope content's disclosure, and audit who has seen ideas that you have initiated.

\paragraph{C5.1:} I have implemented the {\em PitchHub prototype}, a platform that supports the innovation collaboration user stories specified. Built on the popular Ruby on Rails framework and good coding practices this artefact is open to future development.

\paragraph{C5.2:} I have {\em deployed a PitchHub prototype} to Callaghan Innovation for their use. This instance contains the following production configuration: a Passenger application server, caching, SSL, and various security measures suggested by OWASP (e.g. ssh hardening, protection against IP spoofing). Visit \url{http://pitchhub.net} to view this instance.


\paragraph{C5.3:} I have implemented a {\em PitchHub prototype with Shamir's Secret Sharing scheme} integrated. Sensitive information such as Pitch Points, suggestions, and comments is encrypted using this service.

\paragraph{C5.4:} I have implemented a {\em PitchHub prototype with diverse secret keepers} to strengthen the security provided by the Secret Sharing scheme. The prototype supports the following databases: MongoDB, Postgres, MySQL and SQLite. To maintain the \textit{plug-and-play} functionality afforded by MongoDB the configuration required by SQL secret keepers has been automated.

\paragraph{C6:} I have virtualised the {\em PitchHub development environment} using Vagrant and Chef to make installation and setup of PitchHub as easy as possible. This was primarily for the benefit of non-technical stakeholders, but will be equally useful for future maintainers.

\paragraph{C7:} I have automated the {\em PitchHub deployment process} using Capistrano to make host configuration and code deployment as easy as possible. The scripts defined facilitate zero-downtime deployments and also enable deployments to be rolled-back easily. This could be useful in the future should any deployments prove to be defective.
\\
\newline
These contributions are instrumental in fulfilling the design requirements identified in Section \ref{S:designRequirements}. The fulfilment relationship between contributions to requirements is shown in Fig 

\begin{figure}[ht]
    \centering
    \includegraphics[width=0.65\textwidth]{contributions_requirements_map}
    \caption{A mapping of contributions to requirements denoting fulfilment.}
    \label{fig:contribution_requirements_mapping}
\end{figure}

C2, C4, and C5.{\em X} all play roles in the fulfilment of requirements D1, D2.1, and D2.2. The user stories created in D1 capture the behaviours required by D1, D2.1, and D2.2. These user stories were used to drive the design process of the UI (C4) that C5.{\em X} ultimately uses to facilitate these behaviours. The strict adherence to the user stories throughout the design and execution of the project has culminated in artefacts that are faithful in their support of the behaviours required. This fulfilment has been verified by Test \rom{1} and thus requirements D1, D2.1, and D2.2 are concluded as fulfilled.

C5.3 and C5.4 contribute to the fulfilment of requirement D2.3. Integrating the Secret Sharing scheme into the prototype and encrypting commercially sensitive data and IP at the Pitch Point level ensures unconditional security in relation to database breaches (of up to \textit{k}). As described in Section \ref{S:evaluationScope} verification of this component was out of scope of this project. Despite this it is believed that requirement D2.3 has been fulfilled to the extent possible within the scope of this project.

C3, C6 and C7 each play roles in satisfying requirement D3. The architecture (C3) aids extensibility through the tiered architecture style adopted. The virtualised development environment (C6) is key in providing portability, enabling \textit{plug-and-play} functionality for all artefact prototype variations (C5.{\em X}). C7 aids in the extensibility of the deployment process, as new tasks and environment targets can be easily added.

C5.{\em X} have been verified as meeting requirement D4 through Tests \rom{3}-\rom{6} against Nielsen's response time thresholds. Specifically, Tests \rom{3}-Tests \rom{5} verify that requests can be fulfilled within the response-time thresholds given the different Secret Sharing configurations and 



\paragraph{D5: The prototype must support a distributed architecture.}




TODO review in light of design requirements

\section{Future Work}

This section discusses potential extensions to the PitchHub prototype.

\subsection{Recommendation Engine}

A potentially useful feature would be if Pitch Cards could be recommended to users based on their ontological profile. This would encourage collaboration through showing users ideas they are more likely to have an interest in. Offline learning is certainly viable, but an interesting challenge would be integrating the recommender to work with the Secret Sharing service in an online learning approach.

\subsection{Usability Extension/Evaluation}

TODO

\subsection{RealMe Integration}

TODO
