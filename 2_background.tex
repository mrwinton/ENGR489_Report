%!TEX root = proj_report_outline.tex
\chapter{Background into Collaborative Platforms for Innovation}

This chapter aims to explore the related works of collaborative platforms used in the innovation space and also contextualises where in this landscape PitchHub aims to occupy. First, this chapter presents a taxonomy of the primary roles used within the collaborative innovation process. Second, this chapter describes the current collaborative platforms being used in the innovation space and establishes where each stands within the role taxonomy. Third, this chapter concludes with a discussion on the practical limitations that are introduced by being innovation-orientated.

\section{Common Roles in Innovation}

The process of driving an idea from its conceptualisation to its realisation commonly requires a variety of actors who bring together the knowledge, skills and resources required to action its fulfillment. For example, the Apple ][ came to being with Steve Wozniak providing the technical knowledge and skills, Steve Jobs providing the project goals and marketing drive, and Mike Markulla providing the resources to finance it's production \cite{livingston2007founders}. Again and again we see similar stories, where innovation is driven in a collaborative configuration rather than solely by one person. To this point Callaghan Innovation has identified four distinct roles that are embodied by the team within the innovation process:

\begin{itemize}
	\item Challenger
	\item Enabler
	\item Solver
	\item Facilitator
\end{itemize}

These four roles represent the different functions required in an innovative product or service's successful execution. Challengers provide the idea or problem to solved in order to realise a business opporunity. Enablers provide the resources required to action the innovation, this may be in terms of man-power, assets or financing. Solvers provide the answer to the idea or problem presented by the Challenger(s). Facilitators provide the connections to drive the innovation's exection, this may be in terms of connecting other people to the idea, or helping the idea gain reputation. Whether these roles are shared amongst a team or fulfilled by a single person in most cases of innovation these roles are too large for one person to embody them all. To continue with the Apple ][ example, we may catergorise Steve Jobs as the challenger, asking why computers can't serve the consumer market, Steve Wozniak can be seen as an enabler and sover, as he both designed the Apple ][ and built them, and Mike Markulla, can be regarded as an enabler and facilitator, as he financed the production and also lent his reputation to the product. 

\section{An Investigation of Collaborative Platforms}
Naturally, a platform that aims to facilitate collaboration for purposes of innovation at it's empowering an idea in relation to these roles. In this section we explore the current solutions being used to facilitate collaboration and discuss how each works in relation to these roles.
\\
\\
\textbf{IdeaForge}\cite{ideaForge:online} is
\\
\\
\textbf{HunchCruncher} can
\\
\\
\textbf{Assembly} is
\\
\\
\textbf{AngelList}



\section{Practical Limitations of Online Collaboration for Innovation}