%!TEX root = proj_report_outline.tex
\chapter{Experiment Design and Test Bed}
Following the completion and deployment of the prototype, an analysis of the artefacts produced was carried out. The evaluation was designed to verify the successfulness of the project against the requirements identified in Chapter \ref{C:requirements}. To build confidence in the results derived this chapter discusses the scope, assumptions, and limitations as well as the design of the experiments themselves.

\section{Evaluation Scope}
In designing the evaluation the requirements identified in Chapter \ref{C:requirements} were deconstructed to clearly define, where possible, the conditions upon which each requirement can be considered fulfilled. The high-level results of this requirements deconstruction is displayed in the following table:
\\
\begin{adjustwidth}{-1.25cm}{}
\begin{tabular}{ |p{1cm}||p{6cm}|p{10cm}|  }
 \hline
 ID & Requirement & Fulfilment Criteria\\
 \hline
    F1 & Enable the innovation community to collaborate & 1. User stories describing collaboration in the innovation 
  community are specified\newline2. These user stories specified are supported\\
\hline
    F2 & Store sensitive data securely & 1. A security expert verified the system\\
\hline
    F3 & Enable users to scope the disclosure of their content/identity & 1. User stories describing content/identity scoping are specified\newline2. These user stories specified are supported\\
 \hline
    F4 & Be portable and extensible & 1. Discern from requirements\\
 \hline
    NF1 & Be performant & 1. Measure(s) of performance are defined\newline2. These measures are met\\
 \hline
    NF2 & Support a distributed architecture & 1. A distributed system is designed and implemented\\
 \hline
\end{tabular}
\end{adjustwidth}
\vspace{1em}

The breakdown's fulfilment criteria denotes the steps required to satisfactorily mark the requirement as completed, however as seen in requirement F2 and F4 some of these steps require external review. These two requirements are discussed first.
\par
To achieve a satisfactory degree of verification for satisfying requirement F2 would require a level of scrutiny and knowledge possessed only by a security expert. While the integration of the Shamir's Secret Sharing service has been implemented to the best of my ability I believe I am unable to fully explore the security aspect of this evaluation. An evaluation of this requirement would require consideration of the following points: first, the cryptographic modules in the Secret Sharing Service must be verified and second, the security of the storage functionality cannot be considered in isolation. Should any combination of access control, communications, or session management components be compromised it is possible that the data protection component may also be compromised by association. Therefore evaluating whether ``data is stored securely'' must be done within the context of the wider system rather than specifically on the Secret Sharing Service. Verifying the Secret Sharing service's cryptographic implementation would take us further down the rabbit hole. OWASP's application security verification standard \cite{OWASP:online} details the minimum steps: all cryptographic modules fail securely, random number generators apply the appropriate standards, all cryptographic modules are validated against FIPS 140-2 or equivalent, all cryptographic modules operate in their approved mode according to their published security policies. Given the large burden to validate this requirement and the project's limited resources I regard the evaluation of requirement F2 out of scope. Cursory steps I have taken to build confidence in the prototype's security are as follows: Shamir Secret Sharing service, all requests enforce ssl, CSRF tokens to prevent CSRF, escaped user input to prevent XSS, 
white-list firewall set-up, \textit{fstab} modification to prevent shared memory attacks, \textit{sysctl} and host file modification to prevent IP Spoofing, and DenyHosts \cite{DenyH6:online} integrated to prevent SSH attacks.
\par
Assessing whether a system is portable and extensible is an inherently subjective endeavour. The fulfilment criteria for requirement F4 reflects this. A system can be regarded as portable if it works in all environments required of it, and likewise a system can be regarded as extensible given it is able to adapt to all it's future requirements. However, determining all the future environments and requirements of the system is a difficult task. PitchHub has therefore sought to achieve general portability and extensibility. Portability has been defined as ``The ease with which a system or component can be transferred from one hardware or software environment to another'' \cite{mattsson2006software}. As discussed in previous sections PitchHub has employed virtualisation and automation to make such tasks easy as possible. Extensibility is informally defined as ``the ability of the system to tolerate additional features or functionality will little or no required rework of previously developed features or functions'' \cite{Extensibility:online}. PitchHub has been developed with this in mind, the layered architecture and MVC pattern followed mean that both hardware and software components are divided in terms of responsibility. This separation of concerns means that future change request modifications will be isolated to components whose responsibility fits. The inability to strictly quantify or qualify whether requirement F4 is met has led me to regard this requirement out of scope for the purposes of this evaluation. However I believe the spirit of this requirement has been reflected in the system's design.
\par
Requirements F1 and F3 have very similar fulfilment criteria and hence have been grouped together for the evaluation. Through this evaluation we establish that PitchHub has the required functionality to support collaboration within the innovation community. The experiment design and test bed implemented for these functional requirements are described in Section \ref{SS:collaborativeFunctionality}.
\par
Requirement NF1 relies on a simulation based evaluation, where synthetic data is seeded to imitate the load the entire innovation community in New Zealand could have on PitchHub. Requirement NF2 is implicitly satisfied through this experiment as the simulation is deployed on distributed infrastructure. Through this experiment we establish that the PitchHub application has the ability to support the innovation community. The experiment design and test bed implemented for these non-functional requirements are described in Section \ref{SS:performance}.

\section{Assumptions and Limitations}

\section{Test Design}

\subsection{Collaboration Functionality}\label{SS:collaborativeFunctionality}

\subsection{Performance}\label{SS:performance}

Performance is important since a system must fulfil the performance requirements, if not, the system will be of limited use, or not used.