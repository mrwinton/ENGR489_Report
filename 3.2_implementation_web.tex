%!TEX root = proj_report_outline.tex
\chapter{Implementation of the Web Application}
The implementation of the web application in keeping with the iterative approach involved progress on many aspects of the project each week: requirements analysis, requirements validation, design, development, testing, and documentation. This enabled the development of PitchHub to harness the learning made from each previous iteration to help inform the next. I found this to make for an evolutionary approach that also simplified change management. In this chapter aspects of the implementation process and iterative approach are explored in greater detail, such as the Behaviour Driven Development process that was incorporated, the multi-iteration development of the authorisation functionality and my experience with deploying the PitchHub web application.

\section{Behaviour Driven Development}
The development process followed in this project can be regarded as test-centric. On the requirements side weekly reports were used, where that week's meeting points and actions were compiled and e-mailed to be signed off by the client, verifying our mutual understanding. Behaviour Driven Development (BDD) was practised to ensure that this understanding directly (and accurately) translated into the prototype.
\par
This kind of development involves requirements being distilled into specifications, where the behaviour of the code is specified ahead of the implementation, and the implementation then has the onus to fulfil this behaviour. With this approach PitchHub's progress was able to be tracked in each iteration in regards to actual business value delivered. A by-product of this approach has resulted in PitchHub's test-suite relying less on tests on classes and units but on specification. The BDD process has been described as producing what is essentially executable documentation \cite{astels2006new}.
\par


\section{Authorisation}

\section{Virtualised Development Environment}

\section{Deployment}
