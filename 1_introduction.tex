%!TEX root = proj_report_outline.tex
\chapter{Introduction}
The aim of this project is to design, prototype and evaluate PitchHub, an on online collaboration system that empowers the innovation community. This is achieved by engaging the various roles present in the innovation ecosystem, functioning on the notion of scoping/trust and ensuring that all sensitive intellectual property shared is stored in a secure manner. This is an industry project in association with Callaghan Innovation.


\section{Motivation}
The deep integration of digital media within modern society has had a profound effect on how we communicate and collaborate with others. Historically innovation had a great deal to do with proximity. Knowledge of the problem and access to the resources required to solve the problem were the primary limiting factors. In the digital age, the move online has essentially obviated this need for proximity. 

As a result the innovation community has been able to tackle problems regardless of geography. Despite this, Callaghan Innovation that the innovation community is to a large extent still fragmented. There are three primary causes for this.
First, the current solutions used in the innovation space for collaboration either focus on networking around people rather than the ideas \cite{Linkedin:online}\cite{Googlegroups:online} or only facilitate collaboration between certain roles in the innovation community (e.g. investors, inventors, entrepreneurs) \cite{100open:online}\cite{Pledge:online}\cite{Angel:online}\cite{Quirky:online}. Second, contributing intellectual property online requires a significant amount of trust from the users. As once submitted the original poster has limited control over their intellectual property's dissemination. Third, the presence cyber threats is ever-growing \cite{Cybersecurity:online}. Services facilitating innovation need to have security measures in place that protect stored intellectual property from potential malicious access.

\section{Proposed Solution}\label{S:projectObjectives}

A solution to the issues identified above is to use a system that is specifically designed for engaging the innovation community, that enables contributors the ability to control the scope of their intellectual property's dissemination, and that ensures security of this intellectual property. To engage the innovation community it must support all roles within the innovation ecosystem. In doing so PitchHub will be able to facilitate collaboration for a large percentage of the collaboration community. By providing functionality for explicit scoping of intellectual property users may protect themselves from accidental disclosure. Finally, by storing sensitive data via a Threshold scheme users can be assured of unconditional security with regard to malicious access (of up to \textit{k} databases, discussed in Chapter \ref{C:threshholdSecurity}).
The aim of this project has been to design, implement and evaluate this solution, henceforth referred to as PitchHub.

\section{Contributions}

\paragraph{C1:} I have developed an {\em Innovation Role Taxonomy} for classifying existing innovation platforms. The relationship semantics between roles in the taxonomy is denoted as either explicitly supported or implicitly supported. Platforms that facilitate innovation collaboration ideally have explicit support for all roles.

\paragraph{C2:} I have designed a {\em new UI} to encourage collaboration and be visually appealing. Content templates and wire frames were key artefacts produced to guide the final implementation.

\paragraph{C3:} I have designed the {\em PitchHub distributed architecture}. The tiered architecture style adopted supports ease of extension for future development.

\paragraph{C4:} I have developed {\em the first suite of user stories} that capture the interactions of innovation collaboration. These stories specify the following: engagement of all roles within the innovation community, the ability to scope content's disclosure, and audit who has seen ideas that you have initiated.

\paragraph{C5.1:} I have implemented the {\em PitchHub prototype}, a platform that supports the innovation collaboration user stories specified. Built on the popular Ruby on Rails framework and good coding practices this artefact is open to future development.

\paragraph{C5.2:} I have {\em deployed a PitchHub prototype} to Callaghan Innovation for their use. This instance contains the following production configuration: a Passenger application server, caching, SSL, and various security measures suggested by OWASP (e.g. ssh hardening, protection against IP spoofing). Visit \url{http://pitchhub.net} to view this instance.

\paragraph{C5.3:} I have implemented a {\em PitchHub prototype with Shamir's Secret Sharing scheme} integrated. Sensitive information such as Pitch Points, suggestions, and comments is encrypted using this service.

\paragraph{C5.4:} I have implemented a {\em PitchHub prototype with diverse secret keepers} to strengthen the security provided by the Secret Sharing scheme. The prototype supports the following databases: MongoDB, Postgres, MySQL and SQLite. To maintain the \textit{plug-and-play} functionality afforded by MongoDB the configuration required by SQL secret keepers has been automated.

\paragraph{C6:} I have virtualised the {\em PitchHub development environment} using Vagrant and Chef to make installation and setup of PitchHub as easy as possible. This was primarily for the benefit of non-technical stakeholders, but will be equally useful for future maintainers.

\paragraph{C7:} I have automated the {\em PitchHub deployment process} using Capistrano to make host configuration and code deployment as easy as possible. The scripts defined facilitate zero-downtime deployments and also enable deployments to be rolled-back easily.